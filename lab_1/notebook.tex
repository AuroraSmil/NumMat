
% Default to the notebook output style

    


% Inherit from the specified cell style.




    
\documentclass[11pt]{article}

    
    
    \usepackage[T1]{fontenc}
    % Nicer default font (+ math font) than Computer Modern for most use cases
    \usepackage{mathpazo}

    % Basic figure setup, for now with no caption control since it's done
    % automatically by Pandoc (which extracts ![](path) syntax from Markdown).
    \usepackage{graphicx}
    % We will generate all images so they have a width \maxwidth. This means
    % that they will get their normal width if they fit onto the page, but
    % are scaled down if they would overflow the margins.
    \makeatletter
    \def\maxwidth{\ifdim\Gin@nat@width>\linewidth\linewidth
    \else\Gin@nat@width\fi}
    \makeatother
    \let\Oldincludegraphics\includegraphics
    % Set max figure width to be 80% of text width, for now hardcoded.
    \renewcommand{\includegraphics}[1]{\Oldincludegraphics[width=.8\maxwidth]{#1}}
    % Ensure that by default, figures have no caption (until we provide a
    % proper Figure object with a Caption API and a way to capture that
    % in the conversion process - todo).
    \usepackage{caption}
    \DeclareCaptionLabelFormat{nolabel}{}
    \captionsetup{labelformat=nolabel}

    \usepackage{adjustbox} % Used to constrain images to a maximum size 
    \usepackage{xcolor} % Allow colors to be defined
    \usepackage{enumerate} % Needed for markdown enumerations to work
    \usepackage{geometry} % Used to adjust the document margins
    \usepackage{amsmath} % Equations
    \usepackage{amssymb} % Equations
    \usepackage{textcomp} % defines textquotesingle
    % Hack from http://tex.stackexchange.com/a/47451/13684:
    \AtBeginDocument{%
        \def\PYZsq{\textquotesingle}% Upright quotes in Pygmentized code
    }
    \usepackage{upquote} % Upright quotes for verbatim code
    \usepackage{eurosym} % defines \euro
    \usepackage[mathletters]{ucs} % Extended unicode (utf-8) support
    \usepackage[utf8x]{inputenc} % Allow utf-8 characters in the tex document
    \usepackage{fancyvrb} % verbatim replacement that allows latex
    \usepackage{grffile} % extends the file name processing of package graphics 
                         % to support a larger range 
    % The hyperref package gives us a pdf with properly built
    % internal navigation ('pdf bookmarks' for the table of contents,
    % internal cross-reference links, web links for URLs, etc.)
    \usepackage{hyperref}
    \usepackage{longtable} % longtable support required by pandoc >1.10
    \usepackage{booktabs}  % table support for pandoc > 1.12.2
    \usepackage[inline]{enumitem} % IRkernel/repr support (it uses the enumerate* environment)
    \usepackage[normalem]{ulem} % ulem is needed to support strikethroughs (\sout)
                                % normalem makes italics be italics, not underlines
    

    
    
    % Colors for the hyperref package
    \definecolor{urlcolor}{rgb}{0,.145,.698}
    \definecolor{linkcolor}{rgb}{.71,0.21,0.01}
    \definecolor{citecolor}{rgb}{.12,.54,.11}

    % ANSI colors
    \definecolor{ansi-black}{HTML}{3E424D}
    \definecolor{ansi-black-intense}{HTML}{282C36}
    \definecolor{ansi-red}{HTML}{E75C58}
    \definecolor{ansi-red-intense}{HTML}{B22B31}
    \definecolor{ansi-green}{HTML}{00A250}
    \definecolor{ansi-green-intense}{HTML}{007427}
    \definecolor{ansi-yellow}{HTML}{DDB62B}
    \definecolor{ansi-yellow-intense}{HTML}{B27D12}
    \definecolor{ansi-blue}{HTML}{208FFB}
    \definecolor{ansi-blue-intense}{HTML}{0065CA}
    \definecolor{ansi-magenta}{HTML}{D160C4}
    \definecolor{ansi-magenta-intense}{HTML}{A03196}
    \definecolor{ansi-cyan}{HTML}{60C6C8}
    \definecolor{ansi-cyan-intense}{HTML}{258F8F}
    \definecolor{ansi-white}{HTML}{C5C1B4}
    \definecolor{ansi-white-intense}{HTML}{A1A6B2}

    % commands and environments needed by pandoc snippets
    % extracted from the output of `pandoc -s`
    \providecommand{\tightlist}{%
      \setlength{\itemsep}{0pt}\setlength{\parskip}{0pt}}
    \DefineVerbatimEnvironment{Highlighting}{Verbatim}{commandchars=\\\{\}}
    % Add ',fontsize=\small' for more characters per line
    \newenvironment{Shaded}{}{}
    \newcommand{\KeywordTok}[1]{\textcolor[rgb]{0.00,0.44,0.13}{\textbf{{#1}}}}
    \newcommand{\DataTypeTok}[1]{\textcolor[rgb]{0.56,0.13,0.00}{{#1}}}
    \newcommand{\DecValTok}[1]{\textcolor[rgb]{0.25,0.63,0.44}{{#1}}}
    \newcommand{\BaseNTok}[1]{\textcolor[rgb]{0.25,0.63,0.44}{{#1}}}
    \newcommand{\FloatTok}[1]{\textcolor[rgb]{0.25,0.63,0.44}{{#1}}}
    \newcommand{\CharTok}[1]{\textcolor[rgb]{0.25,0.44,0.63}{{#1}}}
    \newcommand{\StringTok}[1]{\textcolor[rgb]{0.25,0.44,0.63}{{#1}}}
    \newcommand{\CommentTok}[1]{\textcolor[rgb]{0.38,0.63,0.69}{\textit{{#1}}}}
    \newcommand{\OtherTok}[1]{\textcolor[rgb]{0.00,0.44,0.13}{{#1}}}
    \newcommand{\AlertTok}[1]{\textcolor[rgb]{1.00,0.00,0.00}{\textbf{{#1}}}}
    \newcommand{\FunctionTok}[1]{\textcolor[rgb]{0.02,0.16,0.49}{{#1}}}
    \newcommand{\RegionMarkerTok}[1]{{#1}}
    \newcommand{\ErrorTok}[1]{\textcolor[rgb]{1.00,0.00,0.00}{\textbf{{#1}}}}
    \newcommand{\NormalTok}[1]{{#1}}
    
    % Additional commands for more recent versions of Pandoc
    \newcommand{\ConstantTok}[1]{\textcolor[rgb]{0.53,0.00,0.00}{{#1}}}
    \newcommand{\SpecialCharTok}[1]{\textcolor[rgb]{0.25,0.44,0.63}{{#1}}}
    \newcommand{\VerbatimStringTok}[1]{\textcolor[rgb]{0.25,0.44,0.63}{{#1}}}
    \newcommand{\SpecialStringTok}[1]{\textcolor[rgb]{0.73,0.40,0.53}{{#1}}}
    \newcommand{\ImportTok}[1]{{#1}}
    \newcommand{\DocumentationTok}[1]{\textcolor[rgb]{0.73,0.13,0.13}{\textit{{#1}}}}
    \newcommand{\AnnotationTok}[1]{\textcolor[rgb]{0.38,0.63,0.69}{\textbf{\textit{{#1}}}}}
    \newcommand{\CommentVarTok}[1]{\textcolor[rgb]{0.38,0.63,0.69}{\textbf{\textit{{#1}}}}}
    \newcommand{\VariableTok}[1]{\textcolor[rgb]{0.10,0.09,0.49}{{#1}}}
    \newcommand{\ControlFlowTok}[1]{\textcolor[rgb]{0.00,0.44,0.13}{\textbf{{#1}}}}
    \newcommand{\OperatorTok}[1]{\textcolor[rgb]{0.40,0.40,0.40}{{#1}}}
    \newcommand{\BuiltInTok}[1]{{#1}}
    \newcommand{\ExtensionTok}[1]{{#1}}
    \newcommand{\PreprocessorTok}[1]{\textcolor[rgb]{0.74,0.48,0.00}{{#1}}}
    \newcommand{\AttributeTok}[1]{\textcolor[rgb]{0.49,0.56,0.16}{{#1}}}
    \newcommand{\InformationTok}[1]{\textcolor[rgb]{0.38,0.63,0.69}{\textbf{\textit{{#1}}}}}
    \newcommand{\WarningTok}[1]{\textcolor[rgb]{0.38,0.63,0.69}{\textbf{\textit{{#1}}}}}
    
    
    % Define a nice break command that doesn't care if a line doesn't already
    % exist.
    \def\br{\hspace*{\fill} \\* }
    % Math Jax compatability definitions
    \def\gt{>}
    \def\lt{<}
    % Document parameters
    \title{lab\_01}
    
    
    

    % Pygments definitions
    
\makeatletter
\def\PY@reset{\let\PY@it=\relax \let\PY@bf=\relax%
    \let\PY@ul=\relax \let\PY@tc=\relax%
    \let\PY@bc=\relax \let\PY@ff=\relax}
\def\PY@tok#1{\csname PY@tok@#1\endcsname}
\def\PY@toks#1+{\ifx\relax#1\empty\else%
    \PY@tok{#1}\expandafter\PY@toks\fi}
\def\PY@do#1{\PY@bc{\PY@tc{\PY@ul{%
    \PY@it{\PY@bf{\PY@ff{#1}}}}}}}
\def\PY#1#2{\PY@reset\PY@toks#1+\relax+\PY@do{#2}}

\expandafter\def\csname PY@tok@w\endcsname{\def\PY@tc##1{\textcolor[rgb]{0.73,0.73,0.73}{##1}}}
\expandafter\def\csname PY@tok@c\endcsname{\let\PY@it=\textit\def\PY@tc##1{\textcolor[rgb]{0.25,0.50,0.50}{##1}}}
\expandafter\def\csname PY@tok@cp\endcsname{\def\PY@tc##1{\textcolor[rgb]{0.74,0.48,0.00}{##1}}}
\expandafter\def\csname PY@tok@k\endcsname{\let\PY@bf=\textbf\def\PY@tc##1{\textcolor[rgb]{0.00,0.50,0.00}{##1}}}
\expandafter\def\csname PY@tok@kp\endcsname{\def\PY@tc##1{\textcolor[rgb]{0.00,0.50,0.00}{##1}}}
\expandafter\def\csname PY@tok@kt\endcsname{\def\PY@tc##1{\textcolor[rgb]{0.69,0.00,0.25}{##1}}}
\expandafter\def\csname PY@tok@o\endcsname{\def\PY@tc##1{\textcolor[rgb]{0.40,0.40,0.40}{##1}}}
\expandafter\def\csname PY@tok@ow\endcsname{\let\PY@bf=\textbf\def\PY@tc##1{\textcolor[rgb]{0.67,0.13,1.00}{##1}}}
\expandafter\def\csname PY@tok@nb\endcsname{\def\PY@tc##1{\textcolor[rgb]{0.00,0.50,0.00}{##1}}}
\expandafter\def\csname PY@tok@nf\endcsname{\def\PY@tc##1{\textcolor[rgb]{0.00,0.00,1.00}{##1}}}
\expandafter\def\csname PY@tok@nc\endcsname{\let\PY@bf=\textbf\def\PY@tc##1{\textcolor[rgb]{0.00,0.00,1.00}{##1}}}
\expandafter\def\csname PY@tok@nn\endcsname{\let\PY@bf=\textbf\def\PY@tc##1{\textcolor[rgb]{0.00,0.00,1.00}{##1}}}
\expandafter\def\csname PY@tok@ne\endcsname{\let\PY@bf=\textbf\def\PY@tc##1{\textcolor[rgb]{0.82,0.25,0.23}{##1}}}
\expandafter\def\csname PY@tok@nv\endcsname{\def\PY@tc##1{\textcolor[rgb]{0.10,0.09,0.49}{##1}}}
\expandafter\def\csname PY@tok@no\endcsname{\def\PY@tc##1{\textcolor[rgb]{0.53,0.00,0.00}{##1}}}
\expandafter\def\csname PY@tok@nl\endcsname{\def\PY@tc##1{\textcolor[rgb]{0.63,0.63,0.00}{##1}}}
\expandafter\def\csname PY@tok@ni\endcsname{\let\PY@bf=\textbf\def\PY@tc##1{\textcolor[rgb]{0.60,0.60,0.60}{##1}}}
\expandafter\def\csname PY@tok@na\endcsname{\def\PY@tc##1{\textcolor[rgb]{0.49,0.56,0.16}{##1}}}
\expandafter\def\csname PY@tok@nt\endcsname{\let\PY@bf=\textbf\def\PY@tc##1{\textcolor[rgb]{0.00,0.50,0.00}{##1}}}
\expandafter\def\csname PY@tok@nd\endcsname{\def\PY@tc##1{\textcolor[rgb]{0.67,0.13,1.00}{##1}}}
\expandafter\def\csname PY@tok@s\endcsname{\def\PY@tc##1{\textcolor[rgb]{0.73,0.13,0.13}{##1}}}
\expandafter\def\csname PY@tok@sd\endcsname{\let\PY@it=\textit\def\PY@tc##1{\textcolor[rgb]{0.73,0.13,0.13}{##1}}}
\expandafter\def\csname PY@tok@si\endcsname{\let\PY@bf=\textbf\def\PY@tc##1{\textcolor[rgb]{0.73,0.40,0.53}{##1}}}
\expandafter\def\csname PY@tok@se\endcsname{\let\PY@bf=\textbf\def\PY@tc##1{\textcolor[rgb]{0.73,0.40,0.13}{##1}}}
\expandafter\def\csname PY@tok@sr\endcsname{\def\PY@tc##1{\textcolor[rgb]{0.73,0.40,0.53}{##1}}}
\expandafter\def\csname PY@tok@ss\endcsname{\def\PY@tc##1{\textcolor[rgb]{0.10,0.09,0.49}{##1}}}
\expandafter\def\csname PY@tok@sx\endcsname{\def\PY@tc##1{\textcolor[rgb]{0.00,0.50,0.00}{##1}}}
\expandafter\def\csname PY@tok@m\endcsname{\def\PY@tc##1{\textcolor[rgb]{0.40,0.40,0.40}{##1}}}
\expandafter\def\csname PY@tok@gh\endcsname{\let\PY@bf=\textbf\def\PY@tc##1{\textcolor[rgb]{0.00,0.00,0.50}{##1}}}
\expandafter\def\csname PY@tok@gu\endcsname{\let\PY@bf=\textbf\def\PY@tc##1{\textcolor[rgb]{0.50,0.00,0.50}{##1}}}
\expandafter\def\csname PY@tok@gd\endcsname{\def\PY@tc##1{\textcolor[rgb]{0.63,0.00,0.00}{##1}}}
\expandafter\def\csname PY@tok@gi\endcsname{\def\PY@tc##1{\textcolor[rgb]{0.00,0.63,0.00}{##1}}}
\expandafter\def\csname PY@tok@gr\endcsname{\def\PY@tc##1{\textcolor[rgb]{1.00,0.00,0.00}{##1}}}
\expandafter\def\csname PY@tok@ge\endcsname{\let\PY@it=\textit}
\expandafter\def\csname PY@tok@gs\endcsname{\let\PY@bf=\textbf}
\expandafter\def\csname PY@tok@gp\endcsname{\let\PY@bf=\textbf\def\PY@tc##1{\textcolor[rgb]{0.00,0.00,0.50}{##1}}}
\expandafter\def\csname PY@tok@go\endcsname{\def\PY@tc##1{\textcolor[rgb]{0.53,0.53,0.53}{##1}}}
\expandafter\def\csname PY@tok@gt\endcsname{\def\PY@tc##1{\textcolor[rgb]{0.00,0.27,0.87}{##1}}}
\expandafter\def\csname PY@tok@err\endcsname{\def\PY@bc##1{\setlength{\fboxsep}{0pt}\fcolorbox[rgb]{1.00,0.00,0.00}{1,1,1}{\strut ##1}}}
\expandafter\def\csname PY@tok@kc\endcsname{\let\PY@bf=\textbf\def\PY@tc##1{\textcolor[rgb]{0.00,0.50,0.00}{##1}}}
\expandafter\def\csname PY@tok@kd\endcsname{\let\PY@bf=\textbf\def\PY@tc##1{\textcolor[rgb]{0.00,0.50,0.00}{##1}}}
\expandafter\def\csname PY@tok@kn\endcsname{\let\PY@bf=\textbf\def\PY@tc##1{\textcolor[rgb]{0.00,0.50,0.00}{##1}}}
\expandafter\def\csname PY@tok@kr\endcsname{\let\PY@bf=\textbf\def\PY@tc##1{\textcolor[rgb]{0.00,0.50,0.00}{##1}}}
\expandafter\def\csname PY@tok@bp\endcsname{\def\PY@tc##1{\textcolor[rgb]{0.00,0.50,0.00}{##1}}}
\expandafter\def\csname PY@tok@fm\endcsname{\def\PY@tc##1{\textcolor[rgb]{0.00,0.00,1.00}{##1}}}
\expandafter\def\csname PY@tok@vc\endcsname{\def\PY@tc##1{\textcolor[rgb]{0.10,0.09,0.49}{##1}}}
\expandafter\def\csname PY@tok@vg\endcsname{\def\PY@tc##1{\textcolor[rgb]{0.10,0.09,0.49}{##1}}}
\expandafter\def\csname PY@tok@vi\endcsname{\def\PY@tc##1{\textcolor[rgb]{0.10,0.09,0.49}{##1}}}
\expandafter\def\csname PY@tok@vm\endcsname{\def\PY@tc##1{\textcolor[rgb]{0.10,0.09,0.49}{##1}}}
\expandafter\def\csname PY@tok@sa\endcsname{\def\PY@tc##1{\textcolor[rgb]{0.73,0.13,0.13}{##1}}}
\expandafter\def\csname PY@tok@sb\endcsname{\def\PY@tc##1{\textcolor[rgb]{0.73,0.13,0.13}{##1}}}
\expandafter\def\csname PY@tok@sc\endcsname{\def\PY@tc##1{\textcolor[rgb]{0.73,0.13,0.13}{##1}}}
\expandafter\def\csname PY@tok@dl\endcsname{\def\PY@tc##1{\textcolor[rgb]{0.73,0.13,0.13}{##1}}}
\expandafter\def\csname PY@tok@s2\endcsname{\def\PY@tc##1{\textcolor[rgb]{0.73,0.13,0.13}{##1}}}
\expandafter\def\csname PY@tok@sh\endcsname{\def\PY@tc##1{\textcolor[rgb]{0.73,0.13,0.13}{##1}}}
\expandafter\def\csname PY@tok@s1\endcsname{\def\PY@tc##1{\textcolor[rgb]{0.73,0.13,0.13}{##1}}}
\expandafter\def\csname PY@tok@mb\endcsname{\def\PY@tc##1{\textcolor[rgb]{0.40,0.40,0.40}{##1}}}
\expandafter\def\csname PY@tok@mf\endcsname{\def\PY@tc##1{\textcolor[rgb]{0.40,0.40,0.40}{##1}}}
\expandafter\def\csname PY@tok@mh\endcsname{\def\PY@tc##1{\textcolor[rgb]{0.40,0.40,0.40}{##1}}}
\expandafter\def\csname PY@tok@mi\endcsname{\def\PY@tc##1{\textcolor[rgb]{0.40,0.40,0.40}{##1}}}
\expandafter\def\csname PY@tok@il\endcsname{\def\PY@tc##1{\textcolor[rgb]{0.40,0.40,0.40}{##1}}}
\expandafter\def\csname PY@tok@mo\endcsname{\def\PY@tc##1{\textcolor[rgb]{0.40,0.40,0.40}{##1}}}
\expandafter\def\csname PY@tok@ch\endcsname{\let\PY@it=\textit\def\PY@tc##1{\textcolor[rgb]{0.25,0.50,0.50}{##1}}}
\expandafter\def\csname PY@tok@cm\endcsname{\let\PY@it=\textit\def\PY@tc##1{\textcolor[rgb]{0.25,0.50,0.50}{##1}}}
\expandafter\def\csname PY@tok@cpf\endcsname{\let\PY@it=\textit\def\PY@tc##1{\textcolor[rgb]{0.25,0.50,0.50}{##1}}}
\expandafter\def\csname PY@tok@c1\endcsname{\let\PY@it=\textit\def\PY@tc##1{\textcolor[rgb]{0.25,0.50,0.50}{##1}}}
\expandafter\def\csname PY@tok@cs\endcsname{\let\PY@it=\textit\def\PY@tc##1{\textcolor[rgb]{0.25,0.50,0.50}{##1}}}

\def\PYZbs{\char`\\}
\def\PYZus{\char`\_}
\def\PYZob{\char`\{}
\def\PYZcb{\char`\}}
\def\PYZca{\char`\^}
\def\PYZam{\char`\&}
\def\PYZlt{\char`\<}
\def\PYZgt{\char`\>}
\def\PYZsh{\char`\#}
\def\PYZpc{\char`\%}
\def\PYZdl{\char`\$}
\def\PYZhy{\char`\-}
\def\PYZsq{\char`\'}
\def\PYZdq{\char`\"}
\def\PYZti{\char`\~}
% for compatibility with earlier versions
\def\PYZat{@}
\def\PYZlb{[}
\def\PYZrb{]}
\makeatother


    % Exact colors from NB
    \definecolor{incolor}{rgb}{0.0, 0.0, 0.5}
    \definecolor{outcolor}{rgb}{0.545, 0.0, 0.0}



    
    % Prevent overflowing lines due to hard-to-break entities
    \sloppy 
    % Setup hyperref package
    \hypersetup{
      breaklinks=true,  % so long urls are correctly broken across lines
      colorlinks=true,
      urlcolor=urlcolor,
      linkcolor=linkcolor,
      citecolor=citecolor,
      }
    % Slightly bigger margins than the latex defaults
    
    \geometry{verbose,tmargin=1in,bmargin=1in,lmargin=1in,rmargin=1in}
    
    

    \begin{document}
    
    
    \maketitle
    
    

    
    \section{Homework 1: Linear Systems Part
I}\label{homework-1-linear-systems-part-i}

    \subsection{General Instructions}\label{general-instructions}

\begin{itemize}
\item
  To pass this assignment requires to complete the present Jupyter
  notebook by

  \begin{itemize}
  \tightlist
  \item
    providing correct answers to all the theoretical exercises, and by
  \item
    providing complete and runnable computer code producing the correct
    results to all the computational problems.
  \end{itemize}
\item
  For the theoretical exercises, please include intermediate steps to
  explain how you arrive at your solution.
\item
  Don't overengineer your code, keep it as simple and readable as
  possible and provide short code comments to help other people
  understanding your code.
\item
  Please provide also a short summary and discussion of your results
  including the requested output (e.g., tables, graphs etc.).
\item
  Up to 3 students can jointly submit the solutions (4 students if at
  least one is an exchange student) \textbf{Only 1 student from each
  group} is supposed to submit them.
\end{itemize}

\subparagraph{\texorpdfstring{Deadline for submission of your solutions
is \textbf{5th of
September}.}{Deadline for submission of your solutions is 5th of September.}}\label{deadline-for-submission-of-your-solutions-is-5th-of-september.}

\subsubsection{Happy coding!}\label{happy-coding}

    \textbf{And before we start:}

Executing the following cell loads a non-default css style for the
notebook. Make sure that you download the corresponding css file
\texttt{tmas4215.css} from the \texttt{lab/styles} Blackbord folder.
Note that the following code snippet assumes that the file resides
inside the folder \texttt{../styles/} relative to the folder where you
stored this notebook on you computer.

    \begin{Verbatim}[commandchars=\\\{\}]
{\color{incolor}In [{\color{incolor}1}]:} \PY{k+kn}{from} \PY{n+nn}{IPython}\PY{n+nn}{.}\PY{n+nn}{core}\PY{n+nn}{.}\PY{n+nn}{display} \PY{k}{import} \PY{n}{HTML}
        \PY{k}{def} \PY{n+nf}{css\PYZus{}styling}\PY{p}{(}\PY{p}{)}\PY{p}{:}
            \PY{n}{styles} \PY{o}{=} \PY{n+nb}{open}\PY{p}{(}\PY{l+s+s2}{\PYZdq{}}\PY{l+s+s2}{../styles/tma4215.css}\PY{l+s+s2}{\PYZdq{}}\PY{p}{,} \PY{l+s+s2}{\PYZdq{}}\PY{l+s+s2}{r}\PY{l+s+s2}{\PYZdq{}}\PY{p}{)}\PY{o}{.}\PY{n}{read}\PY{p}{(}\PY{p}{)}
            \PY{k}{return} \PY{n}{HTML}\PY{p}{(}\PY{n}{styles}\PY{p}{)}
        
        \PY{c+c1}{\PYZsh{} Comment out next line and execute this cell to restore the default notebook style }
        \PY{n}{css\PYZus{}styling}\PY{p}{(}\PY{p}{)}
\end{Verbatim}


\begin{Verbatim}[commandchars=\\\{\}]
{\color{outcolor}Out[{\color{outcolor}1}]:} <IPython.core.display.HTML object>
\end{Verbatim}
            
    \subsubsection{Useful code snippets}\label{useful-code-snippets}

We provide a few of code snippets to get you started in Python. Three
dots \(\ldots\) indicate places where you have to fill in code. We start
with importing the necessary scientific libraries and define a name
alias for them.

    \begin{Verbatim}[commandchars=\\\{\}]
{\color{incolor}In [{\color{incolor}2}]:} \PY{c+c1}{\PYZsh{} Arrary and stuff }
        \PY{k+kn}{import} \PY{n+nn}{numpy} \PY{k}{as} \PY{n+nn}{np}
        \PY{c+c1}{\PYZsh{} Linear algebra solvers from scipy}
        \PY{k+kn}{import} \PY{n+nn}{scipy}\PY{n+nn}{.}\PY{n+nn}{linalg} \PY{k}{as} \PY{n+nn}{la}
        \PY{c+c1}{\PYZsh{} Basic plotting routines from the matplotlib library }
        \PY{k+kn}{import} \PY{n+nn}{matplotlib}\PY{n+nn}{.}\PY{n+nn}{pyplot} \PY{k}{as} \PY{n+nn}{plt}
\end{Verbatim}


    \subsection{Problem 1}\label{problem-1}

Provide a complete proof of Theorem 2 from Lecture 2, following the
outline provided there.

    \textbf{Theorem 2:} Let \(A \in \mathbb{R}^{n,n}\) with \(n \geq2\) and
assume that every leading principal submatrix of order \(k\) with \$1
\leq k \leq n-1 \$ is invertible. Then \(A\) admits a \(LU\)
factorization, where \(L\) is unit lower triangular of order \(n\), and
\(R\) is upper triangular of order \(n\).

\textbf{Proof by induction}

\emph{Base case}: First we will show the theorem for the base case
\(n = 2\). Let \(A^{(2)} = L^{(2)}U^{(2)}\).

\[
\begin{pmatrix} 
a & b \\
c & d 
\end{pmatrix} = 
\begin{pmatrix} 
1 & 0 \\
m & 1 
\end{pmatrix}
\begin{pmatrix} 
u & v \\
0 & \eta
\end{pmatrix} =
\begin{pmatrix} 
u & v \\
mu & mv + \eta
\end{pmatrix}
\]

This gives the four equations \(u = a, v = b, m = c/u = c/a\) and
\(\eta = d - mv = d - c\). So we see the \(LU\)-factorization exists
when \(n = 2\).

\emph{Inductive step} Now assume the theorem holds for all matrices of
order \(k\), \(2 \leq k \leq n\), and suppose that
\(A\in \mathbb{R}^{n+1, n+1}\). Now write \(A, L\) and \(U\) as block
matrices:

\[
A= 
\begin{pmatrix} 
A^{(n)} & \mathbf{b} \\
\mathbf{c}^T & d 
\end{pmatrix}, \quad
L = 
\begin{pmatrix} 
L^{(n)} & \mathbf{0} \\
\mathbf{m}^T & 1 
\end{pmatrix}, \quad
L = 
\begin{pmatrix} 
U^{(n)} & \mathbf{v} \\
\mathbf{0}^T & \eta 
\end{pmatrix}
\]

Then we can write out the product \(A=LU\):

\[
A= 
\begin{pmatrix} 
L^{(n)}U^{(n)} & L^{(n)}\mathbf{v} \\
\mathbf{m}^T U^{(n)} & \mathbf{m}^T \mathbf{v} + \eta
\end{pmatrix}, \quad
\]

This gives the four equations:

\[
A^{(n)} = L^{(n)}U^{(n)} \\
L^{(n)}\mathbf{v} = \mathbf{b} \\
U^{(n)T} \mathbf{m} = \mathbf{c} \\
\mathbf{m}^T \mathbf{v} + \eta = d \\
\]

We know by the assumptions of the theorem that \(A^{(n)}\) is
invertible. Then we know that \(\det(A^{(n)}) \neq 0\) by the invertible
matrix theorem. Since
\(\det(A^{(n)}) = \det(L^{(n)}U^{(n)}) = \det(L^{(n)})\det(U^{(n)}) \neq 0\),
both \(\det(L^{(n)})\) and \(\det(U^{(n)})\) is also have to be
different from zero, and hence both invertible matrices

    \subsection{Problem 2}\label{problem-2}

    Given matrix \(A \in \mathbb{R}^{n,n}\) and
\(\boldsymbol{b} \in \mathbb{R}^n\), the goal of this problem set is to
compute the solution \(\boldsymbol{x}\) to the linear system
\(A \boldsymbol{x} = \boldsymbol{b}\) numerically by implementing the
algorithms \(A = LU\) factorization (whenever possible), and the
backward and forward substitution steps in \texttt{Python}.

As a preliminary step, please make sure that you have import
\texttt{numpy} as \texttt{np} by executing the cell right under "Useful
code snippets".

    \textbf{a)} Implement a \texttt{Python} function \texttt{forward\_sub}:

    \begin{Verbatim}[commandchars=\\\{\}]
{\color{incolor}In [{\color{incolor}3}]:} \PY{k}{def} \PY{n+nf}{forward\PYZus{}sub}\PY{p}{(}\PY{n}{L}\PY{p}{,} \PY{n}{b}\PY{p}{)}\PY{p}{:}
            \PY{n}{n} \PY{o}{=} \PY{n+nb}{len}\PY{p}{(}\PY{n}{b}\PY{p}{)}
            \PY{n}{y}\PY{o}{=}\PY{n}{np}\PY{o}{.}\PY{n}{zeros}\PY{p}{(}\PY{n}{n}\PY{p}{)}
            \PY{k}{for} \PY{n}{i} \PY{o+ow}{in} \PY{n+nb}{range}\PY{p}{(}\PY{n}{n}\PY{p}{)}\PY{p}{:}
                \PY{n}{y\PYZus{}temp} \PY{o}{=} \PY{n}{b}\PY{p}{[}\PY{n}{i}\PY{p}{]}
                \PY{k}{for} \PY{n}{j} \PY{o+ow}{in} \PY{n+nb}{range}\PY{p}{(}\PY{n}{i}\PY{p}{)}\PY{p}{:}
                    \PY{n}{y\PYZus{}temp}\PY{o}{\PYZhy{}}\PY{o}{=} \PY{n}{L}\PY{p}{[}\PY{n}{i}\PY{p}{]}\PY{p}{[}\PY{n}{j}\PY{p}{]}\PY{o}{*}\PY{n}{y}\PY{p}{[}\PY{n}{j}\PY{p}{]}
        
                \PY{n}{y}\PY{p}{[}\PY{n}{i}\PY{p}{]} \PY{o}{=}\PY{n}{y\PYZus{}temp}
            \PY{k}{return} \PY{n}{y}
\end{Verbatim}


    \textbf{b)} Next, implement a \texttt{Python} function
\texttt{backward\_sub}:

    \begin{Verbatim}[commandchars=\\\{\}]
{\color{incolor}In [{\color{incolor}4}]:} \PY{k}{def} \PY{n+nf}{backward\PYZus{}sub}\PY{p}{(}\PY{n}{U}\PY{p}{,} \PY{n}{y}\PY{p}{)}\PY{p}{:}
        
            \PY{c+c1}{\PYZsh{} ...}
            \PY{n}{n} \PY{o}{=} \PY{n+nb}{len}\PY{p}{(}\PY{n}{y}\PY{p}{)}
            \PY{n}{x} \PY{o}{=} \PY{n}{np}\PY{o}{.}\PY{n}{zeros}\PY{p}{(}\PY{n}{n}\PY{p}{)}
            \PY{k}{for} \PY{n}{i} \PY{o+ow}{in} \PY{n+nb}{range}\PY{p}{(}\PY{n}{n}\PY{p}{)}\PY{p}{:}
                \PY{n}{x\PYZus{}temp} \PY{o}{=} \PY{n}{y}\PY{p}{[}\PY{n}{n}\PY{o}{\PYZhy{}}\PY{n}{i}\PY{o}{\PYZhy{}}\PY{l+m+mi}{1}\PY{p}{]}
        
                \PY{k}{for} \PY{n}{j} \PY{o+ow}{in} \PY{n+nb}{range}\PY{p}{(}\PY{n}{i}\PY{p}{)}\PY{p}{:}
                    \PY{n}{x\PYZus{}temp} \PY{o}{\PYZhy{}}\PY{o}{=} \PY{n}{U}\PY{p}{[}\PY{n}{n} \PY{o}{\PYZhy{}} \PY{n}{i}\PY{o}{\PYZhy{}}\PY{l+m+mi}{1}\PY{p}{]}\PY{p}{[}\PY{n}{n} \PY{o}{\PYZhy{}} \PY{n}{j} \PY{o}{\PYZhy{}} \PY{l+m+mi}{1}\PY{p}{]} \PY{o}{*} \PY{n}{x}\PY{p}{[}\PY{n}{n} \PY{o}{\PYZhy{}} \PY{n}{j}\PY{o}{\PYZhy{}}\PY{l+m+mi}{1}\PY{p}{]}
                \PY{n}{x\PYZus{}temp} \PY{o}{=}\PY{n}{x\PYZus{}temp} \PY{o}{/} \PY{n}{U}\PY{p}{[}\PY{n}{n} \PY{o}{\PYZhy{}} \PY{n}{i}\PY{o}{\PYZhy{}}\PY{l+m+mi}{1}\PY{p}{]}\PY{p}{[}\PY{n}{n} \PY{o}{\PYZhy{}} \PY{n}{i}\PY{o}{\PYZhy{}}\PY{l+m+mi}{1}\PY{p}{]}
                \PY{n}{x}\PY{p}{[}\PY{n}{n} \PY{o}{\PYZhy{}} \PY{n}{i} \PY{o}{\PYZhy{}} \PY{l+m+mi}{1}\PY{p}{]} \PY{o}{=} \PY{n}{x\PYZus{}temp}
        
        
            \PY{k}{return} \PY{n}{x}
\end{Verbatim}


    \textbf{c)} Now, implement a Python function which computes for a given
matrix \(A \boldsymbol{R}^{n,n}\) the \(LU\) factorization of \(A = LU\)
((if possible). You can either base your implemenentation on the
Banachiewicz or Crout method (explain in Lecture 3) or any method you
might dig up from the literature. If you want, you can take some
inspiration from the reference \textbf{YEB}, Chapter 3.3, see Program 4,
5, 6.

If the factorization fails without permutating \(A\), you should at
least be so kind and raise an \texttt{Exception}, see
\href{https://docs.python.org/3/tutorial/errors.html\#errors-and-exceptions}{Python
3 tutorial, 8. Errors and Exceptions}. More specifically, you can simply
raise an
\href{https://docs.python.org/3/library/exceptions.html\#NotImplementedError}{NotImplementedError}
exception.

Of course, you can also implement the full \(PA = LU\) decomposition if
you don't want to throw Exceptions around :).

    \begin{Verbatim}[commandchars=\\\{\}]
{\color{incolor}In [{\color{incolor}5}]:} \PY{k}{def} \PY{n+nf}{lu\PYZus{}factor}\PY{p}{(}\PY{n}{A}\PY{p}{)}\PY{p}{:}
            \PY{n}{n} \PY{o}{=} \PY{n+nb}{len}\PY{p}{(}\PY{n}{A}\PY{p}{)}
            \PY{n}{L} \PY{o}{=} \PY{n}{np}\PY{o}{.}\PY{n}{identity}\PY{p}{(}\PY{n}{n}\PY{p}{)}
            \PY{n}{U} \PY{o}{=} \PY{n}{np}\PY{o}{.}\PY{n}{zeros}\PY{p}{(}\PY{n}{A}\PY{o}{.}\PY{n}{shape}\PY{p}{)}
        
        
            \PY{k}{for} \PY{n}{i} \PY{o+ow}{in} \PY{n+nb}{range}\PY{p}{(}\PY{n}{n}\PY{p}{)}\PY{p}{:}
                \PY{k}{if} \PY{n}{A}\PY{p}{[}\PY{n}{i}\PY{p}{,} \PY{n}{i}\PY{p}{]} \PY{o}{==} \PY{l+m+mi}{0}\PY{p}{:}
                    \PY{k}{raise} \PY{n+ne}{Exception}\PY{p}{(}\PY{l+s+s2}{\PYZdq{}}\PY{l+s+s2}{Null pivot element}\PY{l+s+s2}{\PYZdq{}}\PY{p}{)}
                \PY{k}{for} \PY{n}{j} \PY{o+ow}{in} \PY{n+nb}{range}\PY{p}{(}\PY{n}{i}\PY{p}{)}\PY{p}{:}
                    \PY{n}{L}\PY{p}{[}\PY{n}{i}\PY{p}{]}\PY{p}{[}\PY{n}{j}\PY{p}{]} \PY{o}{=} \PY{p}{(}\PY{l+m+mi}{1}\PY{o}{/}\PY{p}{(}\PY{n}{U}\PY{p}{[}\PY{n}{j}\PY{p}{]}\PY{p}{[}\PY{n}{j}\PY{p}{]}\PY{p}{)}\PY{p}{)}\PY{o}{*}\PY{p}{(}\PY{n}{A}\PY{p}{[}\PY{n}{i}\PY{p}{]}\PY{p}{[}\PY{n}{j}\PY{p}{]}\PY{p}{)}
                    \PY{k}{for} \PY{n}{k} \PY{o+ow}{in} \PY{n+nb}{range}\PY{p}{(}\PY{n}{j}\PY{p}{)}\PY{p}{:}
                        \PY{n}{L}\PY{p}{[}\PY{n}{i}\PY{p}{]}\PY{p}{[}\PY{n}{j}\PY{p}{]} \PY{o}{\PYZhy{}}\PY{o}{=} \PY{n}{L}\PY{p}{[}\PY{n}{i}\PY{p}{]}\PY{p}{[}\PY{n}{k}\PY{p}{]}\PY{o}{*}\PY{n}{U}\PY{p}{[}\PY{n}{k}\PY{p}{]}\PY{p}{[}\PY{n}{j}\PY{p}{]}\PY{o}{/}\PY{n}{U}\PY{p}{[}\PY{n}{j}\PY{p}{]}\PY{p}{[}\PY{n}{j}\PY{p}{]}
        
                \PY{k}{for} \PY{n}{j} \PY{o+ow}{in} \PY{n+nb}{range}\PY{p}{(}\PY{n}{i}\PY{p}{,}\PY{n}{n}\PY{p}{)}\PY{p}{:}
                    \PY{n}{U}\PY{p}{[}\PY{n}{i}\PY{p}{]}\PY{p}{[}\PY{n}{j}\PY{p}{]} \PY{o}{=} \PY{n}{A}\PY{p}{[}\PY{n}{i}\PY{p}{]}\PY{p}{[}\PY{n}{j}\PY{p}{]}
                    \PY{k}{for} \PY{n}{k} \PY{o+ow}{in} \PY{n+nb}{range}\PY{p}{(}\PY{n}{i}\PY{p}{)}\PY{p}{:}
                        \PY{n}{U}\PY{p}{[}\PY{n}{i}\PY{p}{]}\PY{p}{[}\PY{n}{j}\PY{p}{]} \PY{o}{\PYZhy{}}\PY{o}{=} \PY{n}{L}\PY{p}{[}\PY{n}{i}\PY{p}{]}\PY{p}{[}\PY{n}{k}\PY{p}{]}\PY{o}{*}\PY{n}{U}\PY{p}{[}\PY{n}{k}\PY{p}{]}\PY{p}{[}\PY{n}{j}\PY{p}{]}
        
            \PY{k}{return} \PY{n}{L}\PY{p}{,}\PY{n}{U}
\end{Verbatim}


    \textbf{d)} Next, combine the Python functions you just implemented in
a)-b) to provide solver for the linear system
\(A \boldsymbol{x} = \boldsymbol{b}\) based on a given \(A = LU\)
factorization.

    \begin{Verbatim}[commandchars=\\\{\}]
{\color{incolor}In [{\color{incolor}6}]:} \PY{k}{def} \PY{n+nf}{lu\PYZus{}solve}\PY{p}{(}\PY{n}{L}\PY{p}{,} \PY{n}{U}\PY{p}{,} \PY{n}{b}\PY{p}{)}\PY{p}{:}
           
            \PY{n}{y} \PY{o}{=} \PY{n}{forward\PYZus{}sub}\PY{p}{(}\PY{n}{L}\PY{p}{,}\PY{n}{b}\PY{p}{)}
            \PY{n}{x} \PY{o}{=} \PY{n}{backward\PYZus{}sub}\PY{p}{(}\PY{n}{U}\PY{p}{,} \PY{n}{y}\PY{p}{)}
            
            \PY{k}{return} \PY{n}{x}
\end{Verbatim}


    \textbf{e)} Finally, write a \texttt{linear\_solve} function by simply
combining your \texttt{lu\_factor} and \texttt{lu\_solve}.

    \begin{Verbatim}[commandchars=\\\{\}]
{\color{incolor}In [{\color{incolor}7}]:} \PY{k}{def} \PY{n+nf}{linear\PYZus{}solve}\PY{p}{(}\PY{n}{A}\PY{p}{,} \PY{n}{b}\PY{p}{)}\PY{p}{:}
            \PY{n}{L}\PY{p}{,} \PY{n}{U} \PY{o}{=} \PY{n}{lu\PYZus{}factor}\PY{p}{(}\PY{n}{A}\PY{p}{)}
            \PY{n}{x} \PY{o}{=} \PY{n}{lu\PYZus{}solve}\PY{p}{(}\PY{n}{L}\PY{p}{,} \PY{n}{U}\PY{p}{,} \PY{n}{b}\PY{p}{)}
            \PY{k}{return} \PY{n}{x}
        
        \PY{l+s+sd}{\PYZdq{}\PYZdq{}\PYZdq{}}
        
        \PY{l+s+sd}{A = np.array([[19, 13,  6, 9, 16],}
        \PY{l+s+sd}{ [ 2, 18,  1,  1,  8],}
        \PY{l+s+sd}{ [12, 17, 12, 10,  6],}
        \PY{l+s+sd}{ [ 5,  4,  6, 18, 11],}
        \PY{l+s+sd}{ [16,  7,  1, 17,  4]])}
        \PY{l+s+sd}{x = np.array([1,1,1,1,1])}
        
        \PY{l+s+sd}{b = np.matmul(A,x)}
        \PY{l+s+sd}{print(b)}
        
        \PY{l+s+sd}{L,U = lu\PYZus{}factor(A)}
        \PY{l+s+sd}{print(L)}
        \PY{l+s+sd}{print(U)}
        \PY{l+s+sd}{print(L@U)}
        \PY{l+s+sd}{print(linear\PYZus{}solve(A,b))}
        \PY{l+s+sd}{\PYZdq{}\PYZdq{}\PYZdq{}}
\end{Verbatim}


\begin{Verbatim}[commandchars=\\\{\}]
{\color{outcolor}Out[{\color{outcolor}7}]:} '\textbackslash{}n\textbackslash{}nA = np.array([[19, 13,  6, 9, 16],\textbackslash{}n [ 2, 18,  1,  1,  8],\textbackslash{}n [12, 17, 12, 10,  6],\textbackslash{}n [ 5,  4,  6, 18, 11],\textbackslash{}n [16,  7,  1, 17,  4]])\textbackslash{}nx = np.array([1,1,1,1,1])\textbackslash{}n\textbackslash{}nb = np.matmul(A,x)\textbackslash{}nprint(b)\textbackslash{}n\textbackslash{}nL,U = lu\_factor(A)\textbackslash{}nprint(L)\textbackslash{}nprint(U)\textbackslash{}nprint(L@U)\textbackslash{}nprint(linear\_solve(A,b))\textbackslash{}n'
\end{Verbatim}
            
    \textbf{f)} Use you brand new \texttt{linear\_solve} function to compute
the solution to a non-trivial linear system. Here non-trivial means that
\(n > 4\) :) and a non-zero right-hand side \(\boldsymbol{b}\).

\emph{Hint}: To check whether your code computes the "exact" result, you
can use the method of \textbf{manufactured solution}. In the case of
linear systems \(A \boldsymbol{x} = \boldsymbol{b}\) this means that for
\textbf{given/chosen} matrix \(A\) and solution vector
\(\boldsymbol{x}\), you simply compute the resulting right-hand side
vector \(\boldsymbol{b} = A \boldsymbol{x}\) to construct an example
where you know the solution. When you now solve
\(A\boldsymbol{x} = \boldsymbol{b}\) starting from \(A\),
\(\boldsymbol{b}\), your implementation should return the correct
\(\boldsymbol{x}\).

    \subsection{Problem 3}\label{problem-3}

The objective of this problem set is two-fold. First, we will have an
actual look at the complexity of the linear solver you implemented in
the previous problem. Second, we want to illustrate that the finite
precision of real numbers can matter very much in actual computations.

We start by introducing the \textbf{Hilbert Matrix} \(H_n\) of order
\(n\) by \[ 
(H_n)_{ij} = \dfrac{1}{i+j-1} \quad \text{for } 1 \leqslant i,j \leqslant n.
\]

The Hilbert matrix is readily available in the \texttt{scipy.linalg}
module, so you simply type, e.g

\begin{Shaded}
\begin{Highlighting}[]
\NormalTok{n }\OperatorTok{=} \DecValTok{3}
\NormalTok{A }\OperatorTok{=}\NormalTok{ la.hilbert(n)}
\end{Highlighting}
\end{Shaded}

Recall that we imported the \texttt{linalg} module and renamed it to
\texttt{la} via the

\begin{Shaded}
\begin{Highlighting}[]
\ImportTok{import}\NormalTok{ scipy.linalg }\ImportTok{as}\NormalTok{ la}
\end{Highlighting}
\end{Shaded}

line at the beginning of this notebook.

    \textbf{a)} Define the trivial righ-hand side
\(\boldsymbol{b} = \boldsymbol{0}\) and measure the executation time for
the various step in your linear\_solver. To do so you can use
\texttt{\%timeit} and \texttt{\%\%timeit} magic functions in IPython,
see
\href{https://ipython.readthedocs.io/en/stable/interactive/magics.html\#magic-timeit}{corresponding
documentation}.

In a nutshell, \texttt{\%\%timeit} measures the executation time of an
entire cell, while \texttt{\%timeit} only measures only the executation
time of a single line, e.g. as in

\begin{Shaded}
\begin{Highlighting}[]
\OperatorTok{%}\NormalTok{timeit my_function()}
\end{Highlighting}
\end{Shaded}

Note that the latter might not play well with functions returning
arguments. To this end, you can simply (re)write a linear solver in 3
lines (1 line per cell) starting from calling your \texttt{lu\_factor}
function and copy-pasting the lines from Step 1 to Step 2 in the
\texttt{linear\_solve()} function. Then use the \texttt{\%\%timeit} to
measure the executation time of each cell.

To get accurate timeing results, \texttt{timeit} automatically runs the
same code multiple times in a loop, and repeats that measurement a
number of times. The actual number of loops and repeats is selected
automatically and will be printed out at the end. They can also be
manually adjusted by using

\begin{verbatim}
%%timeit -n<number_of_loops> -r<number_of_repeats>
\end{verbatim}

Now for \(n = 500, 1000, 2000, 4000\), measure the executation time for
each step in the linear solve (factorizing, forward and backward
substitution) and plot the execution time \(t\) against the number of
unknowns \(n\) in a \(\log\)-\(\log\) plot, that is, \(\log(t)\) against
\(\log(n)\). (Matplotlib has specific functions for log-log plotting)

    \begin{Verbatim}[commandchars=\\\{\}]
{\color{incolor}In [{\color{incolor}8}]:} \PY{n}{n} \PY{o}{=} \PY{l+m+mi}{500}
        \PY{n}{A} \PY{o}{=} \PY{n}{la}\PY{o}{.}\PY{n}{hilbert}\PY{p}{(}\PY{n}{n}\PY{p}{)}
        
        \PY{n}{b} \PY{o}{=} \PY{n}{np}\PY{o}{.}\PY{n}{zeros}\PY{p}{(}\PY{n}{n}\PY{p}{)}
\end{Verbatim}


    \begin{Verbatim}[commandchars=\\\{\}]
{\color{incolor}In [{\color{incolor}9}]:} \PY{o}{\PYZpc{}\PYZpc{}}\PY{k}{timeit} \PYZhy{}n10 \PYZhy{}r1
        L, U = lu\PYZus{}factor(A)
\end{Verbatim}


    \begin{Verbatim}[commandchars=\\\{\}]
40.2 s ± 0 ns per loop (mean ± std. dev. of 1 run, 10 loops each)

    \end{Verbatim}

    \begin{Verbatim}[commandchars=\\\{\}]
{\color{incolor}In [{\color{incolor}10}]:} \PY{n}{L}\PY{p}{,} \PY{n}{U} \PY{o}{=} \PY{n}{lu\PYZus{}factor}\PY{p}{(}\PY{n}{A}\PY{p}{)}
\end{Verbatim}


    \begin{Verbatim}[commandchars=\\\{\}]
{\color{incolor}In [{\color{incolor}11}]:} \PY{o}{\PYZpc{}\PYZpc{}}\PY{k}{timeit}
         y = forward\PYZus{}sub(L, b)
\end{Verbatim}


    \begin{Verbatim}[commandchars=\\\{\}]
44 ms ± 292 µs per loop (mean ± std. dev. of 7 runs, 10 loops each)

    \end{Verbatim}

    \begin{Verbatim}[commandchars=\\\{\}]
{\color{incolor}In [{\color{incolor}12}]:} \PY{n}{y} \PY{o}{=} \PY{n}{forward\PYZus{}sub}\PY{p}{(}\PY{n}{L}\PY{p}{,} \PY{n}{b}\PY{p}{)}
\end{Verbatim}


    \begin{Verbatim}[commandchars=\\\{\}]
{\color{incolor}In [{\color{incolor}13}]:} \PY{o}{\PYZpc{}\PYZpc{}}\PY{k}{timeit}
         x = backward\PYZus{}sub(U, y)
\end{Verbatim}


    \begin{Verbatim}[commandchars=\\\{\}]
59.3 ms ± 324 µs per loop (mean ± std. dev. of 7 runs, 10 loops each)

    \end{Verbatim}

    \textbf{b)} Next, define a non-trivial vector \(\boldsymbol{b}\) by \[
b_i = \sum_{j=1}^n (j/(i+j-1)) 
\] so that the \textbf{exact solution} to the lineary system
\(H_n \boldsymbol{x} = \boldsymbol{b}\) is the vector
\(\boldsymbol{x}_{\mathrm{ex}}\) with elements
\(x_{\mathrm{ex},i} = i\). Now solve the system using your
\texttt{linear\_solve} function for \(n = 2, 4, 8, 16\), print the
computed solution vector \(\boldsymbol{x}_{\mathrm{comp}}\) and compare
it with \(\boldsymbol{x}_{\mathrm{ex}}\). In particular compute

\[\|\boldsymbol{x}_{\mathrm{ex}}-\boldsymbol{x}_{\mathrm{comp}}\|_2.\]

What do you observe? Can you explain your observations?

\emph{Hint}: It can be shown that the condition number \(\kappa_2(H_n)\)
scales like \[
\kappa_2(H_n) \sim \dfrac{\left(\sqrt{2}+1\right)^{4n+4}}{2^{15/4}\sqrt{\pi n}}
\text{as } n \to  \infty.
\]

Tabulate the approximate values of \(\kappa_2(H_n)\) for
\(n = 2, 4, 8, 16\).

This exercise was heavily inspired by Section 2.8 in \textbf{BLUB}.


    % Add a bibliography block to the postdoc
    
    
    
    \end{document}
